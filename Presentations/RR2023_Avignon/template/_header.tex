% Document Class options:
% - You can use article(beamer) or presentation(beamer)
% - In presentation, you can use following options:
%    'trans' for transparent, 'handout' for printing-friendly version
%    'notes' also output notes, 'notesonly' only output notes

% If you have several lectures defined in the same document, uncomment this:
%\includeonlylecture{cours02}

%\usepackage{fontspec}
%\fontfamily{ppl}\selectfont
%\setmainfont{Arial}
%\setmainfont{DejaVu LGC Sans}
%\setmainfont[SmallCapsFont=Fontin SmallCaps]{Fontin-Regular}
%\setmainfont{Gentium Basic}
%\setsansfont{DejaVu LGC Sans}
%\setmonofont{DejaVu LGC Sans Mono}
%\setmainfont{Calibri}
%\setsansfont{Candara}
%\setmonofont{Consolas}

% UMONS-EcoNum template configuration
\mode<all>
{
  \definecolor{umons-red}{RGB}{168, 0, 57}
  \definecolor{umons-turquoise}{RGB}{0, 171, 204}
  \definecolor{umons-gray}{RGB}{150, 150, 150}
  \definecolor{umons-gray2}{RGB}{110, 110, 110}
  \usepackage[labelfont=bf, tableposition=top]{caption}
}
\mode<article>
{
  \usepackage{beamerbasearticle}
  \usepackage{fullpage}
  \pdfimageresolution 72
}
\mode<beamer>
{
  % Customizable items (make sure corresponding pictures are in /template subdir):
  % The background image
  \usebackgroundtemplate{\includegraphics[width=1.07\paperwidth,
height=1.07\paperheight]{template/background-tech.pdf}}
  % The a picture
  \logo{\includegraphics[width=2cm]{template/UMONS-logo.pdf}}
}
\mode<handout>
{
  \definecolor{umons-red}{RGB}{84, 84, 84}
  \definecolor{umons-turquoise}{RGB}{168, 168, 168}
  \usepackage{pgfpages}
  \pgfpagesuselayout{2 on 1}[a4paper, border shrink=5mm]
}
\mode<presentation>
{
  % General theme to use
  \usetheme{CambridgeUS}
  %\usetheme{AnnArbor}
  %\usetheme{default}
  %\usetheme{Pittsburgh}
  %\usetheme{Dresden}
  %\usetheme{Hannover}
  % Color theme (but must colors redefined hereunder)
  \usecolortheme{beaver}
  % Theme used for bullets
  \useinnertheme{rectangles}
  %or \useinnertheme{circles}
  % This is to get semi-transparent covered items
  \setbeamercovered{transparent}
  % Rounded rectangular blocks without shadow
  \setbeamertemplate{blocks}[rounded][shadow=false]
 %\usepackage{beamerthemeshadow}

  \setbeamersize{text margin left=1em,text margin right=1em}
  \setbeamerfont{title}{size=\huge}
  \setbeamerfont{subtitle}{size=\large}
  \setbeamerfont{institute}{size=\small}
  \setbeamerfont{date}{size=\small}
  \setbeamerfont{framesubtitle}{size=\small}
  % This stands out a little bit more than umons-red!
  \setbeamercolor{alerted text}{fg=red!80!black}
  %\setbeamercolor{alerted text}{fg=umons-red}
  \setbeamercolor*{palette primary}{fg=black, bg=umons-turquoise}
  \setbeamercolor*{palette secondary}{fg=black, bg=umons-turquoise}
  \setbeamercolor*{palette tertiary}{fg=umons-gray!30, bg=umons-red!80!black}
  \setbeamercolor*{palette quaternary}{fg=black, bg=umons-turquoise}
  \setbeamercolor*{upper separation line head left}{parent=palette tertiary}
  \setbeamercolor*{upper separation line head right}{parent=palette primary}
  \setbeamercolor{title}{fg=umons-red}
  \setbeamercolor*{titlelike}{fg=umons-red}
  \setbeamercolor{institute}{fg=umons-gray2}
  \setbeamercolor{date}{fg=umons-gray2}
  \setbeamercolor{frametitle}{fg=umons-red, bg=white}
  \setbeamercolor{frametitle right}{bg=white}
  \setbeamercolor{structure}{fg=umons-turquoise}
  \setbeamercolor*{separation line}{}
  \setbeamercolor*{fine separation line}{}
  \setbeamercolor*{sidebar}{fg=umons-red,bg=white}
  \setbeamercolor*{palette sidebar primary}{fg=umons-turquoise!10!black}
  \setbeamercolor*{palette sidebar secondary}{fg=umons-red}
  \setbeamercolor*{palette sidebar tertiary}{fg=umons-red!50!black}
  \setbeamercolor*{palette sidebar quaternary}{fg=umons-red}
  \setbeamercolor*{example text}{fg=umons-gray2}

  % Change the footer
  \makeatother
  \setbeamertemplate{footline}
  {
    \leavevmode%
    \hbox{%
    \begin{beamercolorbox}[wd=.35\paperwidth,ht=2.25ex,dp=1ex,center]{author in head/foot}%
      \usebeamerfont{author in head/foot}\insertshortauthor
    \end{beamercolorbox}%
    \begin{beamercolorbox}[wd=.57\paperwidth,ht=2.25ex,dp=1ex,center]{title in head/foot}%
      \usebeamerfont{title in head/foot}\insertshorttitle\hspace*{3em}
    \end{beamercolorbox}%
    \begin{beamercolorbox}[wd=.08\paperwidth,ht=2.25ex,dp=1ex,center]{title in head/foot}%
      \insertframenumber{} / \inserttotalframenumber\hspace*{1ex}
    \end{beamercolorbox}}%
    \vskip0pt%
  }
  \makeatletter
  \setbeamertemplate{navigation symbols}{}

  % Custom Sweave styles
  %\definecolor{SinputColor}{RGB}{25, 25, 125}
  %\definecolor{SoutputColor}{RGB}{110, 110, 110}
  %\DefineVerbatimEnvironment{Sinput}{Verbatim}{fontsize=\small,formatcom=\color{SinputColor}}
  %\DefineVerbatimEnvironment{Soutput}{Verbatim}{fontsize=\small,formatcom=\color{SoutputColor}}
  %\DefineVerbatimEnvironment{Scode}{Verbatim}{fontsize=\small,\color{SinputColor}}
}

% To get coherent font between text and equations
\usefonttheme{professionalfonts}
%\usepackage{mathpazo}
% Useful package in this context
\usepackage{listings}
% Command used to place a picture somewhere in the slide (relative to current pos)
\newcommand{\putat}[3]{\begin{picture}(0,0)(0,0)\put(#1,#2){#3}\end{picture}}

% Idem, but in absolute pos (slide = 128x96mm)
\usepackage{textpos}
\newcommand{\putabs}[3]{\begin{textblock*}{10mm}[0,0](#1,#2){#3}\end{textblock*}}


% Copyright notices and sources:
% The background picture: http://www.uberpiglet.com/vectors/20-amazing-vector-backgrounds/
% UMONS-EcoNum template is partly inspired from a work done by:
% C. Troestler <Christophe.Troestler@umons.ac.be> licensed under GNU GPL v3 or later.

% SLIDE NOTES:
% For slide notes, use the \note{} beamer style in your document.
% To make sure to generate note pages even where there are no notes
% (in order to have side-by-side slides at left and notes at right),
% we add this:
\makeatletter
\def\beamer@framenotesbegin{%
  \gdef\beamer@noteitems{}%
  \gdef\beamer@notes{{}}%
}
\makeatother
% Uncomment this line to print notes:
%\setbeameroption{show notes}
% To interleave slides and notes in the PDF:
%\setbeamertemplate{note page}[plain]
% To get only notes:
%\setbeameroption{show only notes}

% Uncomment the following 3 lines in presentation mode for dual screen
% use (and specify 'left', 'top', 'right' or 'bottom' for second screen...)
%\usepackage{pgfpages}
%\setbeameroption{second mode text on second screen=right}
%\setbeameroption{show notes on second screen}

% More packages and options
\usepackage{graphicx}
\usepackage{rotating}
%\setbeamertemplate{caption}[numbered]
\usepackage{hyperref}
\usepackage[normalem]{ulem}
%\mode<presentation>
\usepackage{wasysym}
%\usepackage{amsmath}

% TODO: redefine colors for syntax highlighting!

% Tricks for two-columns sections
\def\begincols{\begin{columns}}
\def\begincol{\begin{column}}
\def\endcol{\end{column}}
\def\endcols{\end{columns}}
